\documentclass{article}
\usepackage{fancyhdr}
\usepackage{tabto}
\usepackage{geometry} % Required for adjusting page dimensions and margins
\usepackage{enumitem}
\usepackage[symbol]{footmisc}
\setlist[itemize]{noitemsep, topsep=0pt, after=\vspace{6pt}}
\newlength{\mylen}
\setbox1=\hbox{$\bullet$}\setbox2=\hbox{\tiny$\bullet$}
\setlength{\mylen}{\dimexpr0.5\ht1-0.5\ht2}
\renewcommand\labelitemi{\raisebox{\mylen}{\tiny$\bullet$}}
\newcommand{\return}{\\[6pt]}
\geometry{
	paper=letterpaper, % Paper size, change to letterpaper for US letter size
	top=2cm, % Top margin
	bottom=2.5cm, % Bottom margin
	left=2.5cm, % Left margin
	right=2.5cm, % Right margin
	headheight=14pt, % Header height
	footskip=1.5cm, % Space from the bottom margin to the baseline of the footer
	headsep=1.2cm, % Space from the top margin to the baseline of the header
	%showframe, % Uncomment to show how the type block is set on the page
}
\newcommand{\sect}[1]{
\noindent\large{\textsc{#1}}\\[-6pt]\normalsize{\noindent\rule{\textwidth}{0.5pt}}
}

\begin{document}
\noindent\begin{center}
	{\huge{\textsc{Corbin J. Allison}}}\\
	Manhattan, KS 66502 $ | $ (785) 410-0045 $ | $ calliso@phys.ksu.edu
\end{center}
\sect{Education} \textbf{Kansas State University}, Manhattan, KS\hfill Expected: May 2024 \\
Bachelor of Science in Physics\\ Minors in Mathematics and Computer Science \begin{itemize}
	\item University Honors Program
	\item Cumulative GPA: 3.808/4.000, Physics GPA: 4.000/4.000
	\item Thesis: Sensitivity of Ionization Asymmetries in Chiral Molecules to Molecular Quantum State 
\end{itemize}
\sect{Research Experience}
\textbf{James R. Macdonald Laboratory}, Manhattan, KS\hfill May 2023 - Present\\
{Undergraduate Research Assistant $ | $ Advisor: Dr. Vinod Kumarappan} \begin{itemize}
	\item Experimental study of chiral molecules with the reconstruction of attosecond beating by interference of two-photon transitions (RABBITT) experimental technique
	\item Use of LabVIEW for data acquisition and parallel processing, camera interfacing with BitFlow
	\item Understanding of high vacuum techniques and particle detection with velocity map imaging (VMI) and microchannel plates (MCP)
\end{itemize}
\textbf{Kansas State University Department of Physics}, Manhattan, KS\hfill May 2023 - August 2023\\
{REU Participant $ | $ Advisor: Dr. Loren Greenman} \begin{itemize}
	\item Sensitivity analysis of ionization asymmetries obtained from time-dependent perturbation theory with respect to Hartree-Fock calculations, basis for Bachelor's thesis (in preparation)
	\item Use of Molpro quantum chemistry software for Hartree-Fock calculations, ePolyScat package for computing molecular photoionization
	\item Use of high performance computing on the Beocat Research Cluster at Kansas State University and Perlmutter at the National Energy Research Scientific Computing Center (NERSC) 
\end{itemize}
\textbf{Kansas State University Department of Physics}, Manhattan, KS\hfill June 2022 - Present\\
{Undergraduate Research Assistant $ | $ Advisor: Dr. Loren Greenman} \begin{itemize}
	\item Theoretical study of ionization asymmetries in chiral molecules using time-dependent perturbation theory, control using the RABBITT experimental technique
	\item Internationally collaborative with groups at ETH Z\"{u}rich and Freie Universität Berlin, opportunity to attend and present at the 2023 Gordon Research Conference on Quantum Control of Light and Matter
	\item Use of high performance computing resources from Beocat and NERSC, programming using Fortran 90, Bash, and Python
\end{itemize}
\sect{Teaching Experience}
\textbf{Descriptive Physics} \hfill January 2022 - May 2023\\
Kansas State University $ | $ Primary instructors: Dr. Barbara Fennel \& Dr. Bret Flanders
\begin{itemize}
	\item Two semesters as undergraduate teaching assistant: introduced lab activities to students, graded lab reports, and held office hours to assist students with course materials
	\item One semester as coordinator: set up and tore down lab activities, led meetings to introduce lab activities to other teaching assistants
\end{itemize}
\textbf{Concepts of Physics} \hfill August 2021 - December 2022\\
Kansas State University $ | $ Primary instructor: Dr. JT Laverty
\begin{itemize}
	\item Two semesters as undergraduate teaching assistant and coordinator
	\item Guided students through lab activities, graded lab reports and exams, set up and tore down lab activities\vspace{-12pt}
\end{itemize}
\sect{Additional Work Experience}
\textbf{Heartland Electric, Inc.}, Fort Riley, KS\hfill May 2015 - August 2022\\
Electrician\begin{itemize}
	\item Industrial, commercial, and residential electrical installation, troubleshooting, and repair
	\item Independent work on control systems, including SCADA, variable frequency drives, and motor control systems
\end{itemize}
\sect{Presentations}
%Kansas Honors Connection Conference, Emporia State University, November 2023. Corbin Allison. ``Control of Ionization Asymmetries in Chiral Molecules with Attosecond Techniques." \return
Physics Undergraduate Research Colloquium, Kansas State University, October 2023. Corbin Allison. ``Control of Ionization Asymmetries in Chiral Molecules with Attosecond Techniques." \return
GRC/GRS Quantum Control of Light and Matter, Newport, RI, August 2023. C. Allison, R.E. Goetz, A. Blech, C.P. Koch, and L. Greenman. ``Control of Photoelectron Circular Dichroism with RABBITT" (poster). \return
NSF REU at K-State: Interactions of Matter, Light and Learning, Kansas State University, August 2023. Corbin Allison. ``Calculation of Ionization Asymmetries in Chiral Molecules."\return
\sect{Publications}
\textit{Continuum-electron interferometry for enhancement of photoelectron circular dichroism and measurement of bound, free, and mixed contributions to chiral response}, R.E. Goetz, A. Blech,
C. Allison, C.P. Koch, and L. Greenman, submitted, (2023)\return
\sect{Awards}
University Honors Program Completion Grant \hfill October 2023 \return
James R. Macdonald Memorial Scholarship \hfill August 2023 - May 2024 \return
K-State Dependent/Spouse Grant \hfill August 2023 - May 2024 \return
Cardwell Fund \hfill August 2022 - May 2023 \return
Basil \& Mary Curnutte Scholarship \hfill August 2022 - May 2023 \return
Leo E. Hudiburg Scholarship \hfill August 2021 - May 2022 \return
K-State Dependent/Spouse Grant \hfill August 2017 - May 2018 \return
Foundation Plus Scholarship\hfill August 2017 - May 2018 \return
\sect{Languages and Skills}
\textbf{Languages:} English (native), Spanish (A2) \return
\textbf{Programming languages:} Python, Fortran, C/C++/C\#, Java, Bash, \LaTeX \return
\textbf{Software:} LabVIEW, Molden, Git, MATLAB, Mathematica, Vim, Jupyter \return
\sect{Programming Projects}
\textbf{Image compression and deblurring with SVD}, MATLAB 
\begin{itemize}
	\item Implementation of algorithms to compress and deblur images using singular value decomposition and low-rank approximation
	\item Application to grayscale and color images (RGB and YCbCr) to compare with JPEG compression
\end{itemize} 
\textbf{The Labyrinth}, Java \& XML
\begin{itemize}
	\item Java based video game for Android devices
	\item Demonstrated object-oriented programming, data structuring using HashMaps, and file I/O with objects
	\item Learned about GUI programming for Android devices using XML
\end{itemize}
\newpage
\sect{Relevant Coursework}
\textbf{Physics Courses:}\begin{itemize}
	\item PHYS 694 - Particle Physics\hfill Spring 2024
	\item PHYS 506 - Advanced Physics Laboratory\hfill Spring 2024
	\item PHYS 870 - Introduction to Atomic, Molecular, and Optical Physics\footnote[1]{Graduate course}\hfill Fall 2023
	\item PHYS 709 - Applied Quantum Mechanics\hfill Fall 2023
	\item PHYS 497 - Senior Research in Physics \hfill Fall 2023
	\item PHYS 662 - Introduction to Quantum Mechanics \hfill Spring 2023
	\item PHYS 636 - Physical Measurements Instrumentation \hfill Spring 2023
	\item PHYS 633 - Electromagnetic Fields II \hfill Spring 2023
	\item PHYS 664 - Thermodynamics and Statistical Physics \hfill Fall 2022
	\item PHYS 532 - Electromagnetic Fields I \hfill Fall 2022
	\item PHYS 325 - Physics III, Relativity, and Quantum Physics \hfill Fall 2022
	\item PHYS 522 - Mechanics \hfill Spring 2022
	\item PHYS 214 - Engineering Physics II \hfill Fall 2021
	\item PHYS 213 - Engineering Physics I \hfill Spring 2021\\[-12pt]
\end{itemize}
\textbf{Mathematics Courses:}
\begin{itemize}
	\item MATH 511 - Introduction to Algebraic Systems \hfill Spring 2024
	\item MATH 715 - Applied Mathematics I$ ^* $ \hfill Fall 2023
	\item MATH 515 - Introduction to Linear Algebra\hfill Spring 2023
	\item MATH 221 - Analytic Geometry and Calculus II\hfill Spring 2020
	\item MATH 340 - Elementary Differential Equations\hfill Summer 2020
	\item MATH 222 - Analytic Geometry and Calculus III\hfill Spring 2019
	\item MATH 220 - Analytic Geometry and Calculus I\hfill Fall 2017\\[-12pt]
\end{itemize}
\textbf{Computer Science Courses:}
\begin{itemize}
	\item CIS 301 - Logical Foundations of Programming\hfill Spring 2024
	\item CIS 300 - Data and Program Structures\hfill Fall 2022
	\item CIS 115 - Introduction to Computing Science\hfill Summer 2022
	\item CIS 200 - Programming Fundamentals\hfill Spring 2022
	\item ECE 241 - Introduction to Computer Engineering\hfill Fall 2020\\[0pt]
\end{itemize}

\end{document}
